% Options for packages loaded elsewhere
\PassOptionsToPackage{unicode}{hyperref}
\PassOptionsToPackage{hyphens}{url}
%
\documentclass[
  12pt,
]{article}
\usepackage{amsmath,amssymb}
\usepackage{lmodern}
\usepackage{iftex}
\ifPDFTeX
  \usepackage[T1]{fontenc}
  \usepackage[utf8]{inputenc}
  \usepackage{textcomp} % provide euro and other symbols
\else % if luatex or xetex
  \usepackage{unicode-math}
  \defaultfontfeatures{Scale=MatchLowercase}
  \defaultfontfeatures[\rmfamily]{Ligatures=TeX,Scale=1}
\fi
% Use upquote if available, for straight quotes in verbatim environments
\IfFileExists{upquote.sty}{\usepackage{upquote}}{}
\IfFileExists{microtype.sty}{% use microtype if available
  \usepackage[]{microtype}
  \UseMicrotypeSet[protrusion]{basicmath} % disable protrusion for tt fonts
}{}
\makeatletter
\@ifundefined{KOMAClassName}{% if non-KOMA class
  \IfFileExists{parskip.sty}{%
    \usepackage{parskip}
  }{% else
    \setlength{\parindent}{0pt}
    \setlength{\parskip}{6pt plus 2pt minus 1pt}}
}{% if KOMA class
  \KOMAoptions{parskip=half}}
\makeatother
\usepackage{xcolor}
\usepackage[margin=1in]{geometry}
\usepackage{color}
\usepackage{fancyvrb}
\newcommand{\VerbBar}{|}
\newcommand{\VERB}{\Verb[commandchars=\\\{\}]}
\DefineVerbatimEnvironment{Highlighting}{Verbatim}{commandchars=\\\{\}}
% Add ',fontsize=\small' for more characters per line
\usepackage{framed}
\definecolor{shadecolor}{RGB}{248,248,248}
\newenvironment{Shaded}{\begin{snugshade}}{\end{snugshade}}
\newcommand{\AlertTok}[1]{\textcolor[rgb]{0.94,0.16,0.16}{#1}}
\newcommand{\AnnotationTok}[1]{\textcolor[rgb]{0.56,0.35,0.01}{\textbf{\textit{#1}}}}
\newcommand{\AttributeTok}[1]{\textcolor[rgb]{0.77,0.63,0.00}{#1}}
\newcommand{\BaseNTok}[1]{\textcolor[rgb]{0.00,0.00,0.81}{#1}}
\newcommand{\BuiltInTok}[1]{#1}
\newcommand{\CharTok}[1]{\textcolor[rgb]{0.31,0.60,0.02}{#1}}
\newcommand{\CommentTok}[1]{\textcolor[rgb]{0.56,0.35,0.01}{\textit{#1}}}
\newcommand{\CommentVarTok}[1]{\textcolor[rgb]{0.56,0.35,0.01}{\textbf{\textit{#1}}}}
\newcommand{\ConstantTok}[1]{\textcolor[rgb]{0.00,0.00,0.00}{#1}}
\newcommand{\ControlFlowTok}[1]{\textcolor[rgb]{0.13,0.29,0.53}{\textbf{#1}}}
\newcommand{\DataTypeTok}[1]{\textcolor[rgb]{0.13,0.29,0.53}{#1}}
\newcommand{\DecValTok}[1]{\textcolor[rgb]{0.00,0.00,0.81}{#1}}
\newcommand{\DocumentationTok}[1]{\textcolor[rgb]{0.56,0.35,0.01}{\textbf{\textit{#1}}}}
\newcommand{\ErrorTok}[1]{\textcolor[rgb]{0.64,0.00,0.00}{\textbf{#1}}}
\newcommand{\ExtensionTok}[1]{#1}
\newcommand{\FloatTok}[1]{\textcolor[rgb]{0.00,0.00,0.81}{#1}}
\newcommand{\FunctionTok}[1]{\textcolor[rgb]{0.00,0.00,0.00}{#1}}
\newcommand{\ImportTok}[1]{#1}
\newcommand{\InformationTok}[1]{\textcolor[rgb]{0.56,0.35,0.01}{\textbf{\textit{#1}}}}
\newcommand{\KeywordTok}[1]{\textcolor[rgb]{0.13,0.29,0.53}{\textbf{#1}}}
\newcommand{\NormalTok}[1]{#1}
\newcommand{\OperatorTok}[1]{\textcolor[rgb]{0.81,0.36,0.00}{\textbf{#1}}}
\newcommand{\OtherTok}[1]{\textcolor[rgb]{0.56,0.35,0.01}{#1}}
\newcommand{\PreprocessorTok}[1]{\textcolor[rgb]{0.56,0.35,0.01}{\textit{#1}}}
\newcommand{\RegionMarkerTok}[1]{#1}
\newcommand{\SpecialCharTok}[1]{\textcolor[rgb]{0.00,0.00,0.00}{#1}}
\newcommand{\SpecialStringTok}[1]{\textcolor[rgb]{0.31,0.60,0.02}{#1}}
\newcommand{\StringTok}[1]{\textcolor[rgb]{0.31,0.60,0.02}{#1}}
\newcommand{\VariableTok}[1]{\textcolor[rgb]{0.00,0.00,0.00}{#1}}
\newcommand{\VerbatimStringTok}[1]{\textcolor[rgb]{0.31,0.60,0.02}{#1}}
\newcommand{\WarningTok}[1]{\textcolor[rgb]{0.56,0.35,0.01}{\textbf{\textit{#1}}}}
\usepackage{graphicx}
\makeatletter
\def\maxwidth{\ifdim\Gin@nat@width>\linewidth\linewidth\else\Gin@nat@width\fi}
\def\maxheight{\ifdim\Gin@nat@height>\textheight\textheight\else\Gin@nat@height\fi}
\makeatother
% Scale images if necessary, so that they will not overflow the page
% margins by default, and it is still possible to overwrite the defaults
% using explicit options in \includegraphics[width, height, ...]{}
\setkeys{Gin}{width=\maxwidth,height=\maxheight,keepaspectratio}
% Set default figure placement to htbp
\makeatletter
\def\fps@figure{htbp}
\makeatother
\setlength{\emergencystretch}{3em} % prevent overfull lines
\providecommand{\tightlist}{%
  \setlength{\itemsep}{0pt}\setlength{\parskip}{0pt}}
\setcounter{secnumdepth}{-\maxdimen} % remove section numbering
\ifLuaTeX
  \usepackage{selnolig}  % disable illegal ligatures
\fi
\IfFileExists{bookmark.sty}{\usepackage{bookmark}}{\usepackage{hyperref}}
\IfFileExists{xurl.sty}{\usepackage{xurl}}{} % add URL line breaks if available
\urlstyle{same} % disable monospaced font for URLs
\hypersetup{
  pdftitle={Assignment1-1 Chen},
  pdfauthor={Chloe Chen},
  hidelinks,
  pdfcreator={LaTeX via pandoc}}

\title{Assignment1-1 Chen}
\author{Chloe Chen}
\date{2023-09-08}

\begin{document}
\maketitle

Turn in this assignment as an HTML or PDF file to ELMS. Make sure to
include the R Markdown or Quarto file that was used to generate it. You
should include the questions in your solutions. You may use the qmd file
of the assignment provided to insert your answers.

\hypertarget{git-and-github}{%
\subsection{Git and GitHub}\label{git-and-github}}

\hypertarget{provide-the-link-to-the-github-repo-that-you-used-to-practice-git-from-week-1.-it-should-have}{%
\subsubsection{1) Provide the link to the GitHub repo that you used to
practice git from Week 1. It should
have:}\label{provide-the-link-to-the-github-repo-that-you-used-to-practice-git-from-week-1.-it-should-have}}

\begin{itemize}
\item
  Your name on the README file.
\item
  At least one commit with your name, with a description of what you did
  in that commit.
\item
  \(https://github.com/yqyhhh/Yang-Lou-Chen.git\)
\end{itemize}

======= \#\# Reading Data

Download both the Angell.dta (Stata data format) dataset and the
Angell.txt dataset from this website:
\url{https://stats.idre.ucla.edu/stata/examples/ara/applied-regression-analysis-by-fox-data-files/}

\hypertarget{read-in-the-.dta-version-and-store-in-an-object-called-angell_stata.}{%
\subsubsection{\texorpdfstring{2) Read in the .dta version and store in
an object called
\texttt{angell\_stata}.}{2) Read in the .dta version and store in an object called angell\_stata.}}\label{read-in-the-.dta-version-and-store-in-an-object-called-angell_stata.}}

\textless\textless\textless\textless\textless\textless\textless{} HEAD

\begin{Shaded}
\begin{Highlighting}[]
\FunctionTok{library}\NormalTok{(haven)}
\NormalTok{angell\_stata}\OtherTok{\textless{}{-}}\FunctionTok{read\_dta}\NormalTok{(}\StringTok{"D:/111/SURV727/Yang{-}Lou{-}Chen/angell.dta"}\NormalTok{)}
\FunctionTok{head}\NormalTok{(angell\_stata)}
\end{Highlighting}
\end{Shaded}

\begin{verbatim}
## # A tibble: 6 x 5
##   city       morint ethhet geomob region
##   <chr>       <dbl>  <dbl>  <dbl> <chr> 
## 1 Rochester    19     20.6   15   E     
## 2 Syracuse     17     15.6   20.2 E     
## 3 Worcester    16.4   22.1   13.6 E     
## 4 Erie         16.2   14     14.8 E     
## 5 Milwaukee    15.8   17.4   17.6 MW    
## 6 Bridgeport   15.3   27.9   17.5 E
\end{verbatim}

\hypertarget{read-in-the-.txt-version-and-store-it-in-an-object-called-angell_txt.}{%
\subsubsection{\texorpdfstring{3) Read in the .txt version and store it
in an object called
\texttt{angell\_txt}.}{3) Read in the .txt version and store it in an object called angell\_txt.}}\label{read-in-the-.txt-version-and-store-it-in-an-object-called-angell_txt.}}

\begin{Shaded}
\begin{Highlighting}[]
\NormalTok{angell\_txt}\OtherTok{\textless{}{-}}\FunctionTok{read.table}\NormalTok{(}\StringTok{"https://stats.oarc.ucla.edu/wp{-}content/uploads/2016/02/angell.txt"}\NormalTok{)}
\FunctionTok{head}\NormalTok{(angell\_txt)}
\end{Highlighting}
\end{Shaded}

\begin{verbatim}
##           V1   V2   V3   V4 V5
## 1  Rochester 19.0 20.6 15.0  E
## 2   Syracuse 17.0 15.6 20.2  E
## 3  Worcester 16.4 22.1 13.6  E
## 4       Erie 16.2 14.0 14.8  E
## 5  Milwaukee 15.8 17.4 17.6 MW
## 6 Bridgeport 15.3 27.9 17.5  E
\end{verbatim}

\hypertarget{what-are-the-differences-between-angell_stata-and-angell_txt-are-there-differences-in-the-classes-of-the-individual-columns}{%
\subsubsection{\texorpdfstring{4) What are the differences between
\texttt{angell\_stata} and \texttt{angell\_txt}? Are there differences
in the classes of the individual
columns?}{4) What are the differences between angell\_stata and angell\_txt? Are there differences in the classes of the individual columns?}}\label{what-are-the-differences-between-angell_stata-and-angell_txt-are-there-differences-in-the-classes-of-the-individual-columns}}

\begin{itemize}
\tightlist
\item
  There are certain variable names in angell\_stata, but the column
  names in angell\_txt are simply V1\ldots V5
\end{itemize}

\hypertarget{make-any-updates-necessary-so-that-angell_txt-is-the-same-as-angell_stata.}{%
\subsubsection{\texorpdfstring{5) Make any updates necessary so that
\texttt{angell\_txt} is the same as
\texttt{angell\_stata}.}{5) Make any updates necessary so that angell\_txt is the same as angell\_stata.}}\label{make-any-updates-necessary-so-that-angell_txt-is-the-same-as-angell_stata.}}

\begin{Shaded}
\begin{Highlighting}[]
\FunctionTok{colnames}\NormalTok{(angell\_txt)}\OtherTok{\textless{}{-}}\FunctionTok{c}\NormalTok{(}\StringTok{"city"}\NormalTok{,}\StringTok{"morint"}\NormalTok{,}\StringTok{"ethhet"}\NormalTok{,}\StringTok{"geomob"}\NormalTok{,}\StringTok{"region"}\NormalTok{)}
\FunctionTok{head}\NormalTok{(angell\_txt)}
\end{Highlighting}
\end{Shaded}

\begin{verbatim}
##         city morint ethhet geomob region
## 1  Rochester   19.0   20.6   15.0      E
## 2   Syracuse   17.0   15.6   20.2      E
## 3  Worcester   16.4   22.1   13.6      E
## 4       Erie   16.2   14.0   14.8      E
## 5  Milwaukee   15.8   17.4   17.6     MW
## 6 Bridgeport   15.3   27.9   17.5      E
\end{verbatim}

\hypertarget{describe-the-ethnic-heterogeneity-variable.-use-descriptive-statistics-such-as-mean-median-standard-deviation-etc.-how-does-it-differ-by-region}{%
\subsubsection{6) Describe the Ethnic Heterogeneity variable. Use
descriptive statistics such as mean, median, standard deviation, etc.
How does it differ by
region?}\label{describe-the-ethnic-heterogeneity-variable.-use-descriptive-statistics-such-as-mean-median-standard-deviation-etc.-how-does-it-differ-by-region}}

\begin{Shaded}
\begin{Highlighting}[]
\FunctionTok{mean}\NormalTok{(angell\_stata}\SpecialCharTok{$}\NormalTok{ethhet)}
\end{Highlighting}
\end{Shaded}

\begin{verbatim}
## [1] 31.37209
\end{verbatim}

\begin{Shaded}
\begin{Highlighting}[]
\FunctionTok{median}\NormalTok{(angell\_stata}\SpecialCharTok{$}\NormalTok{ethhet)}
\end{Highlighting}
\end{Shaded}

\begin{verbatim}
## [1] 23.7
\end{verbatim}

\begin{Shaded}
\begin{Highlighting}[]
\FunctionTok{sd}\NormalTok{(angell\_stata}\SpecialCharTok{$}\NormalTok{ethhet)}
\end{Highlighting}
\end{Shaded}

\begin{verbatim}
## [1] 20.41149
\end{verbatim}

\begin{Shaded}
\begin{Highlighting}[]
\FunctionTok{library}\NormalTok{(dplyr)}
\end{Highlighting}
\end{Shaded}

\begin{verbatim}
## 
## 载入程辑包:'dplyr'
\end{verbatim}

\begin{verbatim}
## The following objects are masked from 'package:stats':
## 
##     filter, lag
\end{verbatim}

\begin{verbatim}
## The following objects are masked from 'package:base':
## 
##     intersect, setdiff, setequal, union
\end{verbatim}

\begin{Shaded}
\begin{Highlighting}[]
\NormalTok{angell\_stata}\SpecialCharTok{\%\textgreater{}\%}
  \FunctionTok{group\_by}\NormalTok{(region)}\SpecialCharTok{\%\textgreater{}\%}
  \FunctionTok{summarize}\NormalTok{(}\AttributeTok{m=}\FunctionTok{mean}\NormalTok{(angell\_stata}\SpecialCharTok{$}\NormalTok{ethhet),}
            \AttributeTok{med=}\FunctionTok{median}\NormalTok{(angell\_stata}\SpecialCharTok{$}\NormalTok{ethhet),}
            \AttributeTok{sd=}\FunctionTok{sd}\NormalTok{(angell\_stata}\SpecialCharTok{$}\NormalTok{ethhet))}
\end{Highlighting}
\end{Shaded}

\begin{verbatim}
## # A tibble: 4 x 4
##   region     m   med    sd
##   <chr>  <dbl> <dbl> <dbl>
## 1 E       31.4  23.7  20.4
## 2 MW      31.4  23.7  20.4
## 3 S       31.4  23.7  20.4
## 4 W       31.4  23.7  20.4
\end{verbatim}

\hypertarget{describing-data}{%
\subsection{Describing Data}\label{describing-data}}

R comes also with many built-in datasets. The ``MASS'' package, for
example, comes with the ``Boston'' dataset.

\hypertarget{install-the-mass-package-load-the-package.-then-load-the-boston-dataset.}{%
\subsubsection{7) Install the ``MASS'' package, load the package. Then,
load the Boston
dataset.}\label{install-the-mass-package-load-the-package.-then-load-the-boston-dataset.}}

\textless\textless\textless\textless\textless\textless\textless{} HEAD

\begin{Shaded}
\begin{Highlighting}[]
\FunctionTok{library}\NormalTok{(MASS)}
\end{Highlighting}
\end{Shaded}

\begin{verbatim}
## 
## 载入程辑包:'MASS'
\end{verbatim}

\begin{verbatim}
## The following object is masked from 'package:dplyr':
## 
##     select
\end{verbatim}

\begin{Shaded}
\begin{Highlighting}[]
\FunctionTok{data}\NormalTok{(Boston)}
\end{Highlighting}
\end{Shaded}

\hypertarget{what-is-the-type-of-the-boston-object}{%
\subsubsection{8) What is the type of the Boston
object?}\label{what-is-the-type-of-the-boston-object}}

\begin{Shaded}
\begin{Highlighting}[]
\FunctionTok{head}\NormalTok{(Boston)}
\end{Highlighting}
\end{Shaded}

\begin{verbatim}
##      crim zn indus chas   nox    rm  age    dis rad tax ptratio  black lstat
## 1 0.00632 18  2.31    0 0.538 6.575 65.2 4.0900   1 296    15.3 396.90  4.98
## 2 0.02731  0  7.07    0 0.469 6.421 78.9 4.9671   2 242    17.8 396.90  9.14
## 3 0.02729  0  7.07    0 0.469 7.185 61.1 4.9671   2 242    17.8 392.83  4.03
## 4 0.03237  0  2.18    0 0.458 6.998 45.8 6.0622   3 222    18.7 394.63  2.94
## 5 0.06905  0  2.18    0 0.458 7.147 54.2 6.0622   3 222    18.7 396.90  5.33
## 6 0.02985  0  2.18    0 0.458 6.430 58.7 6.0622   3 222    18.7 394.12  5.21
##   medv
## 1 24.0
## 2 21.6
## 3 34.7
## 4 33.4
## 5 36.2
## 6 28.7
\end{verbatim}

\begin{Shaded}
\begin{Highlighting}[]
\FunctionTok{typeof}\NormalTok{(Boston)}
\end{Highlighting}
\end{Shaded}

\begin{verbatim}
## [1] "list"
\end{verbatim}

\begin{itemize}
\tightlist
\item
  The type of the object is a list.
\end{itemize}

\hypertarget{what-is-the-class-of-the-boston-object}{%
\subsubsection{9) What is the class of the Boston
object?}\label{what-is-the-class-of-the-boston-object}}

\begin{Shaded}
\begin{Highlighting}[]
\FunctionTok{class}\NormalTok{(Boston)}
\end{Highlighting}
\end{Shaded}

\begin{verbatim}
## [1] "data.frame"
\end{verbatim}

\begin{itemize}
\tightlist
\item
  The class id data frame.
\end{itemize}

\hypertarget{how-many-of-the-suburbs-in-the-boston-data-set-bound-the-charles-river}{%
\subsubsection{10) How many of the suburbs in the Boston data set bound
the Charles
river?}\label{how-many-of-the-suburbs-in-the-boston-data-set-bound-the-charles-river}}

\begin{Shaded}
\begin{Highlighting}[]
\FunctionTok{summary}\NormalTok{(Boston}\SpecialCharTok{$}\NormalTok{chas)}
\end{Highlighting}
\end{Shaded}

\begin{verbatim}
##    Min. 1st Qu.  Median    Mean 3rd Qu.    Max. 
## 0.00000 0.00000 0.00000 0.06917 0.00000 1.00000
\end{verbatim}

\begin{Shaded}
\begin{Highlighting}[]
\NormalTok{nsub}\OtherTok{\textless{}{-}}\DecValTok{506}\SpecialCharTok{*}\FunctionTok{mean}\NormalTok{(Boston}\SpecialCharTok{$}\NormalTok{chas)}
\NormalTok{nsub}
\end{Highlighting}
\end{Shaded}

\begin{verbatim}
## [1] 35
\end{verbatim}

\begin{itemize}
\tightlist
\item
  There \(35\) suburbs set bound the Charles river.
\end{itemize}

\hypertarget{do-any-of-the-suburbs-of-boston-appear-to-have-particularly-high-crime-rates-tax-rates-pupil-teacher-ratios-comment-on-the-range-of-each-variable.}{%
\subsubsection{11) Do any of the suburbs of Boston appear to have
particularly high crime rates? Tax rates? Pupil-teacher ratios? Comment
on the range of each
variable.}\label{do-any-of-the-suburbs-of-boston-appear-to-have-particularly-high-crime-rates-tax-rates-pupil-teacher-ratios-comment-on-the-range-of-each-variable.}}

\begin{Shaded}
\begin{Highlighting}[]
\NormalTok{range\_crim }\OtherTok{\textless{}{-}} \FunctionTok{range}\NormalTok{(Boston}\SpecialCharTok{$}\NormalTok{crim)}
\FunctionTok{cat}\NormalTok{(}\StringTok{"Range of crim:"}\NormalTok{, range\_crim, }\StringTok{"}\SpecialCharTok{\textbackslash{}n}\StringTok{"}\NormalTok{)}
\end{Highlighting}
\end{Shaded}

\begin{verbatim}
## Range of crim: 0.00632 88.9762
\end{verbatim}

\begin{Shaded}
\begin{Highlighting}[]
\NormalTok{range\_tax }\OtherTok{\textless{}{-}} \FunctionTok{range}\NormalTok{(Boston}\SpecialCharTok{$}\NormalTok{tax)}
\FunctionTok{cat}\NormalTok{(}\StringTok{"Range of tax:"}\NormalTok{, range\_tax, }\StringTok{"}\SpecialCharTok{\textbackslash{}n}\StringTok{"}\NormalTok{)}
\end{Highlighting}
\end{Shaded}

\begin{verbatim}
## Range of tax: 187 711
\end{verbatim}

\begin{Shaded}
\begin{Highlighting}[]
\NormalTok{range\_ptratio }\OtherTok{\textless{}{-}} \FunctionTok{range}\NormalTok{(Boston}\SpecialCharTok{$}\NormalTok{ptratio)}
\FunctionTok{cat}\NormalTok{(}\StringTok{"Range of ptratio:"}\NormalTok{, range\_ptratio, }\StringTok{"}\SpecialCharTok{\textbackslash{}n}\StringTok{"}\NormalTok{)}
\end{Highlighting}
\end{Shaded}

\begin{verbatim}
## Range of ptratio: 12.6 22
\end{verbatim}

\begin{Shaded}
\begin{Highlighting}[]
\FunctionTok{boxplot}\NormalTok{(Boston}\SpecialCharTok{$}\NormalTok{crim, }\AttributeTok{main=}\StringTok{"Boxplot of crim"}\NormalTok{, }
        \AttributeTok{xlab=}\StringTok{"crim"}\NormalTok{, }\AttributeTok{ylab=}\StringTok{"Values"}\NormalTok{,}
        \AttributeTok{col=}\FunctionTok{c}\NormalTok{(}\StringTok{"blue"}\NormalTok{))}
\end{Highlighting}
\end{Shaded}

\includegraphics{Assignment1-1-Chen_files/figure-latex/unnamed-chunk-9-1.pdf}

\begin{Shaded}
\begin{Highlighting}[]
\FunctionTok{boxplot}\NormalTok{(Boston}\SpecialCharTok{$}\NormalTok{tax, }\AttributeTok{main=}\StringTok{"Boxplot of tax"}\NormalTok{, }
        \AttributeTok{xlab=}\StringTok{"tax"}\NormalTok{, }\AttributeTok{ylab=}\StringTok{"Values"}\NormalTok{,}
        \AttributeTok{col=}\FunctionTok{c}\NormalTok{(}\StringTok{"green"}\NormalTok{))}
\end{Highlighting}
\end{Shaded}

\includegraphics{Assignment1-1-Chen_files/figure-latex/unnamed-chunk-9-2.pdf}

\begin{Shaded}
\begin{Highlighting}[]
\FunctionTok{boxplot}\NormalTok{(Boston}\SpecialCharTok{$}\NormalTok{tax, }\AttributeTok{main=}\StringTok{"Boxplot of ptratio"}\NormalTok{, }
        \AttributeTok{xlab=}\StringTok{"ptratio"}\NormalTok{, }\AttributeTok{ylab=}\StringTok{"Values"}\NormalTok{,}
        \AttributeTok{col=}\FunctionTok{c}\NormalTok{(}\StringTok{"red"}\NormalTok{))}
\end{Highlighting}
\end{Shaded}

\includegraphics{Assignment1-1-Chen_files/figure-latex/unnamed-chunk-9-3.pdf}

\begin{itemize}
\tightlist
\item
  The range of crime is \([0.00632, 88.9762]\); the range of tax is
  \([187, 711]\); the range of ptratio is \([12.6, 22]\).
\item
  As we can see from the ranges and box-plots, there are particularly
  high crime rates in several suburbs.
\item
  That doesn't exist in the other two viriables. The range of tax is
  larger than ptratio, but there are no outliers occur. ptratio is the
  most compactly distributed data, with little difference between
  suburbs on this variable.
\end{itemize}

\hypertarget{describe-the-distribution-of-pupil-teacher-ratio-among-the-towns-in-this-data-set-that-have-a-per-capita-crime-rate-larger-than-1.-how-does-it-differ-from-towns-that-have-a-per-capita-crime-rate-smaller-than-1}{%
\subsubsection{12) Describe the distribution of pupil-teacher ratio
among the towns in this data set that have a per capita crime rate
larger than 1. How does it differ from towns that have a per capita
crime rate smaller than
1?}\label{describe-the-distribution-of-pupil-teacher-ratio-among-the-towns-in-this-data-set-that-have-a-per-capita-crime-rate-larger-than-1.-how-does-it-differ-from-towns-that-have-a-per-capita-crime-rate-smaller-than-1}}

\begin{Shaded}
\begin{Highlighting}[]
\NormalTok{subset1 }\OtherTok{\textless{}{-}} \FunctionTok{subset}\NormalTok{(Boston, Boston}\SpecialCharTok{$}\NormalTok{crim }\SpecialCharTok{\textgreater{}} \DecValTok{1}\NormalTok{)}
\NormalTok{subset2 }\OtherTok{\textless{}{-}} \FunctionTok{subset}\NormalTok{(Boston, Boston}\SpecialCharTok{$}\NormalTok{crim }\SpecialCharTok{\textless{}=} \DecValTok{1}\NormalTok{)}

\CommentTok{\#summary}
\FunctionTok{summary}\NormalTok{(subset1}\SpecialCharTok{$}\NormalTok{ptratio)}
\end{Highlighting}
\end{Shaded}

\begin{verbatim}
##    Min. 1st Qu.  Median    Mean 3rd Qu.    Max. 
##   14.70   20.20   20.20   19.29   20.20   21.20
\end{verbatim}

\begin{Shaded}
\begin{Highlighting}[]
\FunctionTok{summary}\NormalTok{(subset2}\SpecialCharTok{$}\NormalTok{ptratio)}
\end{Highlighting}
\end{Shaded}

\begin{verbatim}
##    Min. 1st Qu.  Median    Mean 3rd Qu.    Max. 
##   12.60   16.80   18.30   18.02   19.20   22.00
\end{verbatim}

\begin{Shaded}
\begin{Highlighting}[]
\FunctionTok{var}\NormalTok{(subset1}\SpecialCharTok{$}\NormalTok{ptratio)}
\end{Highlighting}
\end{Shaded}

\begin{verbatim}
## [1] 4.484129
\end{verbatim}

\begin{Shaded}
\begin{Highlighting}[]
\FunctionTok{var}\NormalTok{(subset2}\SpecialCharTok{$}\NormalTok{ptratio)}
\end{Highlighting}
\end{Shaded}

\begin{verbatim}
## [1] 4.243581
\end{verbatim}

\begin{Shaded}
\begin{Highlighting}[]
\CommentTok{\#Box plot}
\FunctionTok{boxplot}\NormalTok{(subset1}\SpecialCharTok{$}\NormalTok{ptratio)}
\end{Highlighting}
\end{Shaded}

\includegraphics{Assignment1-1-Chen_files/figure-latex/unnamed-chunk-10-1.pdf}

\begin{Shaded}
\begin{Highlighting}[]
\FunctionTok{boxplot}\NormalTok{(subset2}\SpecialCharTok{$}\NormalTok{ptratio)}
\end{Highlighting}
\end{Shaded}

\includegraphics{Assignment1-1-Chen_files/figure-latex/unnamed-chunk-10-2.pdf}

\begin{Shaded}
\begin{Highlighting}[]
\CommentTok{\#Histogram}
\FunctionTok{hist}\NormalTok{(subset1}\SpecialCharTok{$}\NormalTok{ptratio)}
\end{Highlighting}
\end{Shaded}

\includegraphics{Assignment1-1-Chen_files/figure-latex/unnamed-chunk-10-3.pdf}

\begin{Shaded}
\begin{Highlighting}[]
\FunctionTok{hist}\NormalTok{(subset2}\SpecialCharTok{$}\NormalTok{ptratio)}

\CommentTok{\#Q{-}Q plot \& Conduct statistical tests}
\FunctionTok{library}\NormalTok{(ggplot2)}
\end{Highlighting}
\end{Shaded}

\includegraphics{Assignment1-1-Chen_files/figure-latex/unnamed-chunk-10-4.pdf}

\begin{Shaded}
\begin{Highlighting}[]
\FunctionTok{qqnorm}\NormalTok{(subset1}\SpecialCharTok{$}\NormalTok{ptratio, }\AttributeTok{main=}\StringTok{"ptratio1"}\NormalTok{, }\AttributeTok{ylab=}\StringTok{"y\_\{i:n\}"}\NormalTok{, }\AttributeTok{xlab=}\StringTok{"m\_\{i:n\}"}\NormalTok{) }
\FunctionTok{qqline}\NormalTok{(subset1}\SpecialCharTok{$}\NormalTok{ptratio, }\AttributeTok{col=}\StringTok{"red"}\NormalTok{,}\AttributeTok{lwd=}\DecValTok{2}\NormalTok{)}
\end{Highlighting}
\end{Shaded}

\includegraphics{Assignment1-1-Chen_files/figure-latex/unnamed-chunk-10-5.pdf}

\begin{Shaded}
\begin{Highlighting}[]
\FunctionTok{qqnorm}\NormalTok{(subset2}\SpecialCharTok{$}\NormalTok{ptratio, }\AttributeTok{main=}\StringTok{"ptratio2"}\NormalTok{, }\AttributeTok{ylab=}\StringTok{"y\_\{i:n\}"}\NormalTok{, }\AttributeTok{xlab=}\StringTok{"m\_\{i:n\}"}\NormalTok{) }
\FunctionTok{qqline}\NormalTok{(subset2}\SpecialCharTok{$}\NormalTok{ptratio, }\AttributeTok{col=}\StringTok{"blue"}\NormalTok{,}\AttributeTok{lwd=}\DecValTok{2}\NormalTok{)}
\end{Highlighting}
\end{Shaded}

\includegraphics{Assignment1-1-Chen_files/figure-latex/unnamed-chunk-10-6.pdf}

\begin{Shaded}
\begin{Highlighting}[]
\FunctionTok{shapiro.test}\NormalTok{(subset1}\SpecialCharTok{$}\NormalTok{ptratio)}
\end{Highlighting}
\end{Shaded}

\begin{verbatim}
## 
##  Shapiro-Wilk normality test
## 
## data:  subset1$ptratio
## W = 0.51756, p-value < 2.2e-16
\end{verbatim}

\begin{Shaded}
\begin{Highlighting}[]
\FunctionTok{shapiro.test}\NormalTok{(subset2}\SpecialCharTok{$}\NormalTok{ptratio)}
\end{Highlighting}
\end{Shaded}

\begin{verbatim}
## 
##  Shapiro-Wilk normality test
## 
## data:  subset2$ptratio
## W = 0.96226, p-value = 1.453e-07
\end{verbatim}

\emph{Measures of Central Tendency} The pupil-teacher ratio among the
towns in this data set that have a per capita crime rate larger than 1
has a mean of \(19.29\) and a median of \(20.20\). \emph{Measures of
Dispersion} The pupil-teacher ratio among the towns in this data set
that have a per capita crime rate larger than 1 has a variance of
\(4.484129\). \emph{Distribution Shape} In Shapiro-Wilk normality test,
p-value \textless{} 0.05, and from the histogram, as well as the
qq-plot, homeprice does not appear to follow a normal distribution.
\emph{Outliers} From the box plot we can see that there are \(4\)
outliers in the distribution.

\hypertarget{writing-functions}{%
\subsection{Writing Functions}\label{writing-functions}}

\hypertarget{write-a-function-that-calculates-95-confidence-intervals-for-a-point-estimate.-the-function-should-be-called-my_ci.-when-called-with-my_ci2-0.2-the-function-should-print-out-the-95-ci-upper-bound-of-point-estimate-2-with-standard-error-0.2-is-2.392.-the-lower-bound-is-1.608.}{%
\subsubsection{\texorpdfstring{13) Write a function that calculates 95\%
confidence intervals for a point estimate. The function should be called
\texttt{my\_CI}. When called with \texttt{my\_CI(2,\ 0.2)}, the function
should print out ``The 95\% CI upper bound of point estimate 2 with
standard error 0.2 is 2.392. The lower bound is
1.608.''}{13) Write a function that calculates 95\% confidence intervals for a point estimate. The function should be called my\_CI. When called with my\_CI(2, 0.2), the function should print out ``The 95\% CI upper bound of point estimate 2 with standard error 0.2 is 2.392. The lower bound is 1.608.''}}\label{write-a-function-that-calculates-95-confidence-intervals-for-a-point-estimate.-the-function-should-be-called-my_ci.-when-called-with-my_ci2-0.2-the-function-should-print-out-the-95-ci-upper-bound-of-point-estimate-2-with-standard-error-0.2-is-2.392.-the-lower-bound-is-1.608.}}

\begin{Shaded}
\begin{Highlighting}[]
\NormalTok{my\_CI}\OtherTok{\textless{}{-}} \ControlFlowTok{function}\NormalTok{(point\_estimate,se)\{}
\NormalTok{  lower\_bound}\OtherTok{\textless{}{-}}\NormalTok{point\_estimate}\FloatTok{{-}1.96}\SpecialCharTok{*}\NormalTok{se}
\NormalTok{  upper\_bound}\OtherTok{\textless{}{-}}\NormalTok{point\_estimate}\FloatTok{+1.96}\SpecialCharTok{*}\NormalTok{se}
\NormalTok{  text }\OtherTok{\textless{}{-}} \FunctionTok{paste}\NormalTok{(}\StringTok{"The 95\% CI upper bound of point estimate"}\NormalTok{, point\_estimate, }\StringTok{"with standard error"}\NormalTok{, se,}\StringTok{"is"}\NormalTok{, upper\_bound, }\StringTok{". The lower bound is"}\NormalTok{, lower\_bound)}
\NormalTok{  text}
\NormalTok{\}}

\NormalTok{ci}\OtherTok{\textless{}{-}}\FunctionTok{my\_CI}\NormalTok{(}\DecValTok{2}\NormalTok{,}\FloatTok{0.2}\NormalTok{)}
\NormalTok{ci}
\end{Highlighting}
\end{Shaded}

\begin{verbatim}
## [1] "The 95% CI upper bound of point estimate 2 with standard error 0.2 is 2.392 . The lower bound is 1.608"
\end{verbatim}

\emph{Note: The function should take a point estimate and its standard
error as arguments. You may use the formula for 95\% CI: point estimate
+/- 1.96*standard error.}

\emph{Note: The function should take a point estimate and its standard
error as arguments. You may use the formula for 95\% CI: point estimate
+/- 1.96*standard error.}

\emph{Hint: Pasting text in R can be done with:} \texttt{paste()}
\emph{and} \texttt{paste0()}

\hypertarget{create-a-new-function-called-my_ci2-that-does-that-same-thing-as-the-my_ci-function-but-outputs-a-vector-of-length-2-with-the-lower-and-upper-bound-of-the-confidence-interval-instead-of-printing-out-the-text.-use-this-to-find-the-95-confidence-interval-for-a-point-estimate-of-0-and-standard-error-0.4.}{%
\subsubsection{\texorpdfstring{14) Create a new function called
\texttt{my\_CI2} that does that same thing as the \texttt{my\_CI}
function but outputs a vector of length 2 with the lower and upper bound
of the confidence interval instead of printing out the text. Use this to
find the 95\% confidence interval for a point estimate of 0 and standard
error
0.4.}{14) Create a new function called my\_CI2 that does that same thing as the my\_CI function but outputs a vector of length 2 with the lower and upper bound of the confidence interval instead of printing out the text. Use this to find the 95\% confidence interval for a point estimate of 0 and standard error 0.4.}}\label{create-a-new-function-called-my_ci2-that-does-that-same-thing-as-the-my_ci-function-but-outputs-a-vector-of-length-2-with-the-lower-and-upper-bound-of-the-confidence-interval-instead-of-printing-out-the-text.-use-this-to-find-the-95-confidence-interval-for-a-point-estimate-of-0-and-standard-error-0.4.}}

\begin{Shaded}
\begin{Highlighting}[]
\NormalTok{my\_CI2}\OtherTok{\textless{}{-}} \ControlFlowTok{function}\NormalTok{(point\_estimate,se)\{}
\NormalTok{  lower\_bound}\OtherTok{\textless{}{-}}\NormalTok{point\_estimate}\FloatTok{{-}1.96}\SpecialCharTok{*}\NormalTok{se}
\NormalTok{  upper\_bound}\OtherTok{\textless{}{-}}\NormalTok{point\_estimate}\FloatTok{+1.96}\SpecialCharTok{*}\NormalTok{se}
  \FunctionTok{c}\NormalTok{(lower\_bound,upper\_bound)}
\NormalTok{\}}

\NormalTok{ci}\OtherTok{\textless{}{-}}\FunctionTok{my\_CI2}\NormalTok{(}\DecValTok{0}\NormalTok{,}\FloatTok{0.4}\NormalTok{)}
\NormalTok{ci}
\end{Highlighting}
\end{Shaded}

\begin{verbatim}
## [1] -0.784  0.784
\end{verbatim}

\hypertarget{update-the-my_ci2-function-to-take-any-confidence-level-instead-of-only-95.-call-the-new-function-my_ci3.-you-should-add-an-argument-to-your-function-for-confidence-level.}{%
\subsubsection{\texorpdfstring{15) Update the \texttt{my\_CI2} function
to take any confidence level instead of only 95\%. Call the new function
\texttt{my\_CI3}. You should add an argument to your function for
confidence
level.}{15) Update the my\_CI2 function to take any confidence level instead of only 95\%. Call the new function my\_CI3. You should add an argument to your function for confidence level.}}\label{update-the-my_ci2-function-to-take-any-confidence-level-instead-of-only-95.-call-the-new-function-my_ci3.-you-should-add-an-argument-to-your-function-for-confidence-level.}}

\begin{Shaded}
\begin{Highlighting}[]
\NormalTok{my\_CI3 }\OtherTok{\textless{}{-}} \ControlFlowTok{function}\NormalTok{(point\_estimate, se, confidence\_level) \{}
  \ControlFlowTok{if}\NormalTok{ (confidence\_level }\SpecialCharTok{\textless{}=} \DecValTok{0} \SpecialCharTok{||}\NormalTok{ confidence\_level }\SpecialCharTok{\textgreater{}=} \DecValTok{1}\NormalTok{) \{}
    \FunctionTok{stop}\NormalTok{(}\StringTok{"Confidence level must be between 0 and 1"}\NormalTok{)}
\NormalTok{  \}}
  
\NormalTok{  z\_value }\OtherTok{\textless{}{-}} \FunctionTok{qnorm}\NormalTok{(}\DecValTok{1} \SpecialCharTok{{-}}\NormalTok{ (}\DecValTok{1} \SpecialCharTok{{-}}\NormalTok{ confidence\_level) }\SpecialCharTok{/} \DecValTok{2}\NormalTok{)}
  
\NormalTok{  lower\_bound }\OtherTok{\textless{}{-}}\NormalTok{ point\_estimate }\SpecialCharTok{{-}}\NormalTok{ z\_value }\SpecialCharTok{*}\NormalTok{ se}
\NormalTok{  upper\_bound }\OtherTok{\textless{}{-}}\NormalTok{ point\_estimate }\SpecialCharTok{+}\NormalTok{ z\_value }\SpecialCharTok{*}\NormalTok{ se}
  
  \FunctionTok{c}\NormalTok{(lower\_bound, upper\_bound)}
\NormalTok{\}}

\NormalTok{ci\_90 }\OtherTok{\textless{}{-}} \FunctionTok{my\_CI3}\NormalTok{(}\DecValTok{0}\NormalTok{, }\FloatTok{0.4}\NormalTok{, }\FloatTok{0.90}\NormalTok{)}
\FunctionTok{print}\NormalTok{(ci\_90)}
\end{Highlighting}
\end{Shaded}

\begin{verbatim}
## [1] -0.6579415  0.6579415
\end{verbatim}

\begin{Shaded}
\begin{Highlighting}[]
\NormalTok{ci\_99 }\OtherTok{\textless{}{-}} \FunctionTok{my\_CI3}\NormalTok{(}\DecValTok{0}\NormalTok{, }\FloatTok{0.4}\NormalTok{, }\FloatTok{0.99}\NormalTok{)}
\FunctionTok{print}\NormalTok{(ci\_99)}
\end{Highlighting}
\end{Shaded}

\begin{verbatim}
## [1] -1.030332  1.030332
\end{verbatim}

\emph{Hint: Use the} \texttt{qnorm} \emph{function to find the
appropriate z-value. For example, for a 95\% confidence interval, using}
\texttt{qnorm(0.975)} \emph{gives approximately 1.96.}

\hypertarget{without-hardcoding-any-numbers-in-the-code-find-a-99-confidence-interval-for-ethnic-heterogeneity-in-the-angell-dataset.-find-the-standard-error-by-dividing-the-standard-deviation-by-the-square-root-of-the-sample-size.}{%
\subsubsection{16) Without hardcoding any numbers in the code, find a
99\% confidence interval for Ethnic Heterogeneity in the Angell dataset.
Find the standard error by dividing the standard deviation by the square
root of the sample
size.}\label{without-hardcoding-any-numbers-in-the-code-find-a-99-confidence-interval-for-ethnic-heterogeneity-in-the-angell-dataset.-find-the-standard-error-by-dividing-the-standard-deviation-by-the-square-root-of-the-sample-size.}}

\textless\textless\textless\textless\textless\textless\textless{} HEAD

\begin{Shaded}
\begin{Highlighting}[]
\NormalTok{se\_ethhet }\OtherTok{\textless{}{-}} \FunctionTok{sd}\NormalTok{(angell\_stata}\SpecialCharTok{$}\NormalTok{ethhet)}\SpecialCharTok{/}\FunctionTok{sqrt}\NormalTok{(}\FunctionTok{nrow}\NormalTok{(angell\_stata))}
\NormalTok{mean\_ethhet }\OtherTok{\textless{}{-}} \FunctionTok{mean}\NormalTok{(angell\_stata}\SpecialCharTok{$}\NormalTok{ethhet)}
\NormalTok{ethhetCI}\OtherTok{\textless{}{-}}\FunctionTok{my\_CI3}\NormalTok{(mean\_ethhet, se\_ethhet, }\FloatTok{0.99}\NormalTok{)}
\NormalTok{ethhetCI}
\end{Highlighting}
\end{Shaded}

\begin{verbatim}
## [1] 23.35425 39.38993
\end{verbatim}

\begin{itemize}
\tightlist
\item
  The 99\% confidence interval for Ethnic Heterogeneity is
  \([23.35425, 39.38993]\).
\end{itemize}

\hypertarget{write-a-function-that-you-can-apply-to-the-angell-dataset-to-get-95-confidence-intervals.-the-function-should-take-one-argument-a-vector.-use-if-else-statements-to-output-na-and-avoid-error-messages-if-the-column-in-the-data-frame-is-not-numeric-or-logical.}{%
\subsubsection{\texorpdfstring{17) Write a function that you can
\texttt{apply} to the Angell dataset to get 95\% confidence intervals.
The function should take one argument: a vector. Use if-else statements
to output NA and avoid error messages if the column in the data frame is
not numeric or
logical.}{17) Write a function that you can apply to the Angell dataset to get 95\% confidence intervals. The function should take one argument: a vector. Use if-else statements to output NA and avoid error messages if the column in the data frame is not numeric or logical.}}\label{write-a-function-that-you-can-apply-to-the-angell-dataset-to-get-95-confidence-intervals.-the-function-should-take-one-argument-a-vector.-use-if-else-statements-to-output-na-and-avoid-error-messages-if-the-column-in-the-data-frame-is-not-numeric-or-logical.}}

\begin{Shaded}
\begin{Highlighting}[]
\NormalTok{my\_CI4 }\OtherTok{\textless{}{-}} \ControlFlowTok{function}\NormalTok{(column) \{}
  \ControlFlowTok{if}\NormalTok{ (}\FunctionTok{is.numeric}\NormalTok{(column) }\SpecialCharTok{|}\FunctionTok{is.logical}\NormalTok{(column)) \{}
\NormalTok{    mean\_value }\OtherTok{\textless{}{-}} \FunctionTok{mean}\NormalTok{(column, }\AttributeTok{na.rm =} \ConstantTok{TRUE}\NormalTok{)}
\NormalTok{    se }\OtherTok{\textless{}{-}} \FunctionTok{sqrt}\NormalTok{(}\FunctionTok{var}\NormalTok{(column, }\AttributeTok{na.rm =} \ConstantTok{TRUE}\NormalTok{) }\SpecialCharTok{/} \FunctionTok{length}\NormalTok{(column))}
\NormalTok{    z\_value }\OtherTok{\textless{}{-}} \FunctionTok{qnorm}\NormalTok{(}\FloatTok{0.975}\NormalTok{)  }\CommentTok{\# 95\% confidence interval}
    
\NormalTok{    lower\_bound }\OtherTok{\textless{}{-}}\NormalTok{ mean\_value }\SpecialCharTok{{-}}\NormalTok{ z\_value }\SpecialCharTok{*}\NormalTok{ se}
\NormalTok{    upper\_bound }\OtherTok{\textless{}{-}}\NormalTok{ mean\_value }\SpecialCharTok{+}\NormalTok{ z\_value }\SpecialCharTok{*}\NormalTok{ se}
    
    \FunctionTok{return}\NormalTok{(}\FunctionTok{c}\NormalTok{(lower\_bound, upper\_bound))}
\NormalTok{  \} }\ControlFlowTok{else}\NormalTok{ \{}
    \FunctionTok{return}\NormalTok{(}\ConstantTok{NA}\NormalTok{)}
\NormalTok{  \}}
\NormalTok{\}}


\NormalTok{result }\OtherTok{\textless{}{-}} \FunctionTok{apply}\NormalTok{(angell\_stata, }\DecValTok{2}\NormalTok{, my\_CI4) }\DocumentationTok{\#\# Apply this function to each column of Agell}

\NormalTok{result}
\end{Highlighting}
\end{Shaded}

\begin{verbatim}
##   city morint ethhet geomob region 
##     NA     NA     NA     NA     NA
\end{verbatim}

\begin{Shaded}
\begin{Highlighting}[]
\FunctionTok{is.numeric}\NormalTok{(angell\_stata}\SpecialCharTok{$}\NormalTok{morint)}
\end{Highlighting}
\end{Shaded}

\begin{verbatim}
## [1] TRUE
\end{verbatim}

\end{document}
